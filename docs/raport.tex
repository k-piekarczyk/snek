\documentclass[12pt]{article}
\usepackage{polski}
\usepackage[utf8]{inputenc}
\title{Raport projektu "snek"}
\author{Krzysztof Piekarczyk, Karol Brzeziński}
\date{25.05.2019}
\begin{document}
\maketitle


\section{Cel projektu}

Gra "snek", która jest implementacją popularnej gry "Snake"

\section{Teoria}

Gracz steruje wężem poruszającym się po planszy o określonej wielkości, uważając aby nie wpaść głową na resztę swojego ciała, gdyż zakończy to grę. W trakcie poruszania się po planszy wąż zjada pojawiające się losowo na mapie jedzenie, które powoduje przedłużenie się jego ciała i zwiększa liczbę uzyskanych punktów.

\section{Możliwości gry}

Gra umożliwia zresetowanie stanu w dowolnej chwili i pokazuje aktualny wynik gracza zależny od długości węża. W przypadku porażki gra wyświetla komunikat informujący o przegranej.

\section{Sterowanie}

Sterowanie wężem odbywa się za pomocą strzałek. Klawisz R doprowadza do zresetowania gry, a ESC do wyjścia.

\section{Informacje o grze}

Początkowa długość węża to 5. Punkty liczone są metodą: długość węża * 5. Gra działa w 25 klatkach na sekundę

\section{Opis klas}

\subsection{\textbf{\textit{Game}}}

Klasa Game odpowiada za przeprowadzenie gry, tworzy ramę, maluje komponenty. Implementuje interfejs KeyListener i dziedziczy JPanel.
\begin{itemize}
	\item \textit{private Game(int width, int height)} - ustawia wielkość
	\item \textit{private void drawPoint(Graphics g, Point point)} - rysuje punkt na planszy
	\item \textit{public void paint(Graphics g)} - odpowiada za malowanie całej planszy w ramie  i wszystkie rzeczy które się na niej znajdują
	\item \textit{private static void run()} - tworzy ramę gry, znajduje się też pętla w której przeprowadzana jest gra
	\item \textit{public void keyPressed(KeyEvent e)} - wczytywanie lub wyjście z gry
	\item \textit{public static void main(String[] args)}
\end{itemize}
\subsection{\textbf{\textit{GameInfo}}}

Znajdująca się w klasie Game, odpowiada z punktację.
\begin{itemize}
	\item \textit{GameInfo(int width)} - ustawia  wielkość
	\item \textit{public void paint(Graphics g)} - maluje odpowiednie elementy
	\item \textit{public void setScore(int score)} - ustawia aktualną punktację 
\end{itemize}
\subsection{\textbf{\textit{Grid}}}

Klasa Grid odpowiada za planszę gry, przemieszczanie się po niej węża i pojawianie jedzenia.
\begin{itemize}
	\item \textit{public Grid(int width, int height)} - ustawia liczbę wierszy i kolumn, początkowe miejsce węża i jedzenia
	\item \textit{private Point getRandomPoint()} - zwraca losowy punkt na mapie
	\item \textit{public Point wrap(Point point)} - odpowiada za  przechodzenie węża przez ściany
	\item \textit{public boolean update()} - w przypadku zjedzenia wywołuje metodę przedłużająca węża i ustawia odpowiedni punkt, wywołuje też ruch i sprawdza czy nie nastąpiła kolizja
	\item \textit{public Snake getSnake()} - zwraca węża
	\item \textit{public Food getFood()} - zwraca jedzenie
	\item \textit{public int getRealX(Point point)} - zwraca współrzędną  x
	\item \textit{public int getRealY(Point point)} - zwraca współrzędną y
\end{itemize}
\subsection{\textbf{\textit{Snake}}}

Klasa Snake odpowiada za samego węża, jego ruch, kierunek ruchu, przedłużenie ciała i sprawdzanie kolizji. Implementuje interfejs KeyListener.
\begin{itemize}
	\item \textit{public Snake(Grid grid, Point point)} - tworzy na planszy ciało węża o odpowiedniej długości
	\item \textit{public Point getHead()} - zwraca głowę węża
	\item \textit{public List<Point> getBody()} - zwraca ciało węża
	\item \textit{public void extend()} - odpowiada za przedłużanie ciała
	\item \textit{public void move()} - ruch węża po mapie
	\item \textit{public int getSize()} - zwraca aktualną wielkość węża
	\item \textit{public boolean didCollide()} - sprawdza czy głowa węża nie zderzyła się z resztą jego ciała
	\item \textit{public void keyPressed(KeyEvent e)} - zwraca odpowiedni kierunek ruchu, który został wywołany naciśnięciem klawiszy
	\end{itemize}

\subsection{\textbf{\textit{Food}}}

Klasa Food odpowiada za pojawiające się na mapie jedzenie. Przy pomocy metody Random wybiera jeden z trzech dostępnych kolorów jako który pojawi się jedzenie.

\begin{itemize}
            \item \textit{public Food(Point point)} - przyjmuje punkt, nadaje mu wybrany losowy kolor
            \item \textit{public Point getPoint()} - zwraca punkt
            \item \textit{public void setPoint(Point point))} - ustawia kolor
            \item \textit{public Color getColor()} - zwraca kolor
\end{itemize}

\end{document}

\documentclass[12pt]{article}
\usepackage{polski}
\usepackage[utf8]{inputenc}
\title{Raport projektu "snek"}
\author{Krzysztof Piekarczyk, Karol Brzeziński}
\date{25.05.2019}
\begin{document}
\maketitle


\section{Cel projektu}

Gra "snek", która jest implementacją popularnej gry "Snake"

\section{Teoria}

Gracz steruje wężem poruszającym się po planszy o określonej wielkości, uważając aby nie wpaść głową na resztę swojego ciała, gdyż zakończy to grę. W trakcie poruszania się po planszy wąż zjada pojawiające się losowo na mapie jedzenie, które powoduje przedłużenie się jego ciała i zwiększa liczbę uzyskanych punktów.

\section{Możliwości gry}

Gra umożliwia zresetowanie stanu w dowolnej chwili i pokazuje aktualny wynik gracza zależny od długości węża. W przypadku porażki gra wyświetla komunikat informujący o przegranej.

\section{Sterowanie}

Sterowanie wężem odbywa się za pomocą strzałek. Klawisz R doprowadza do zresetowania gry, a ESC do wyjścia.

\section{Opis klas}

\subsection{\textbf{\textit{Game}}}

\subsection{\textbf{\textit{Grid}}}

\subsection{\textbf{\textit{Snake}}}

\subsection{\textbf{\textit{Food}}}


\end{document}
